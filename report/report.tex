\documentclass[conference]{IEEEtran}
\usepackage{cite}
\usepackage{amsmath,amssymb,amsfonts}
\usepackage{algorithmic}
\usepackage{graphicx}
\usepackage{textcomp}
\usepackage{xcolor}
\begin{document}

\title{Using the Jason Belief-Desire-Intention framework for implementing RoboCup soccer-playing agents\\
}

\author{\IEEEauthorblockN{Josh Blondin}
\IEEEauthorblockA{\textit{Systems and Computer Engineering} \\
\textit{Carleton University}\\
Ottawa, Canada \\
joshblondin@cmail.carleton.ca}
\and
\IEEEauthorblockN{Thomas Ross}
\IEEEauthorblockA{\textit{Systems and Computer Engineering} \\
\textit{Carleton University}\\
Ottawa, Canada \\
thomasross@cmail.carleton.ca}
\and
\IEEEauthorblockN{Nicholas Sendyk}
\IEEEauthorblockA{\textit{Systems and Computer Engineering} \\
\textit{Carleton University}\\
Ottawa, Canada \\
nicksendyk@cmail.carleton.ca}
\and
\IEEEauthorblockN{William Tran}
\IEEEauthorblockA{\textit{Systems and Computer Engineering} \\
\textit{Carleton University}\\
Ottawa, Canada \\
williamtran5@cmail.carleton.ca}
}

\maketitle

\begin{abstract}
(J.B.) Jason is a popular AgentSpeak framework used to build and execute BDI-based agents.
In particular, we have looked into its use in developing software agents for RoboCup, an international soccer framework where the competitors are agents.
The goal of the project and this paper was to explore how to use Jason to create these agents, not necessarily to create optimized agents.
For this purpose we created three agents with different goals, desires, and movement patterns, to showcase how different agents acting individually would operate together.
\end{abstract}

\begin{IEEEkeywords}
keyword1, keyword2
\end{IEEEkeywords}

\section{Introduction}
Introduction goes here. Example citation~\cite{b1}.

\section{Integration with Jason}
\subsection{Running the Code}

\section{Agents}
\subsection{Offensive agent (J.B.)}
The offensive agent utilizes predicates in order to perceive the ball, enemy goal and self goal.
Using the predicate for the ball, there are rules to see of the agent is facing the ball or at the ball are used.
There are sub-goals, with the initial sub-goal being to find the ball.
These sub-goals are what the agent desires, and then the intention of its actions are chosen from these desires.
The main goal of an offensive agent is to score a goal.
In order to achieve this, the offensive agent which was implemented follows a simple pattern: first it finds the ball, then it runs to the ball, then it finds the goal, and finally it kicks the ball to the goal.
This is a very simple implementation, as the design goal of the agent was to prove the usability of Jason's BDI framework rather than making an optimized agent.
There are still however small optimizations made which show how Jason can be used to optimize agents to make incredibly smart decisions.
When the agent dashes to the ball, it constantly turns until it is facing the ball and then only dashes.
This greatly increases its accuracy so that it won't miss the ball while maintaining a high speed.
One other example of an optimization is when the agent kicks the ball, if it does not see the enemy goal and does see its own goal, it will kick the ball in the opposite direction of its own goal.
This greatly reduces the time it would take to find the goal in which an enemy could kick the ball away, and assures that the ball will always be kicked in an appropriate direction.

\subsection{Defensive agent}
\subsection{Goalkeeping agent}

\section{Testing}

\section{Future Work (J.B.)}
As the aim of this project was to showcase and utilize the Jason BDI framework in the context of RoboCup, there was not much effort put in to optimizing the agents' playstyles.
Future work with this project would be to optimize these agents to better play and better represent their positions.
One such major optimization would be increasing stamina efficiency, especially on the offensive player which is almost always dashing.
The position of the goals relative to the agents could also always be tracked when the agent cannot view them in order to avoid having to spend extra time turning.
Non-goalie agents could also only stay in specifc parts of the field in order to avoid moving back and forth, and the defensive agent could be made to have more accurate passes.
All of these optimizations are only examples of what the team discovered, and examples of what could be worked on using this codebase in the future, however there are most certianly many others which could also be worked on.

\section{Conclusion}

\begin{thebibliography}{00}
\bibitem{b1} G. Eason, B. Noble, and I. N. Sneddon, ``On certain integrals of Lipschitz-Hankel type involving products of Bessel functions,'' Phil. Trans. Roy. Soc. London, vol. A247, pp. 529--551, April 1955.
\bibitem{b2} J. Clerk Maxwell, A Treatise on Electricity and Magnetism, 3rd ed., vol. 2. Oxford: Clarendon, 1892, pp.68--73.
\bibitem{b3} I. S. Jacobs and C. P. Bean, ``Fine particles, thin films and exchange anisotropy,'' in Magnetism, vol. III, G. T. Rado and H. Suhl, Eds. New York: Academic, 1963, pp. 271--350.
\bibitem{b4} K. Elissa, ``Title of paper if known,'' unpublished.
\bibitem{b5} R. Nicole, ``Title of paper with only first word capitalized,'' J. Name Stand. Abbrev., in press.
\bibitem{b6} Y. Yorozu, M. Hirano, K. Oka, and Y. Tagawa, ``Electron spectroscopy studies on magneto-optical media and plastic substrate interface,'' IEEE Transl. J. Magn. Japan, vol. 2, pp. 740--741, August 1987 [Digests 9th Annual Conf. Magnetics Japan, p. 301, 1982].
\bibitem{b7} M. Young, The Technical Writer's Handbook. Mill Valley, CA: University Science, 1989.
\end{thebibliography}

\end{document}
